\documentclass[8pt]{beamer}
\usepackage{amsmath}
\usepackage{amssymb}
\usepackage{graphicx}
\usepackage{hyperref}
\usepackage{color}
\usepackage{float}
\usepackage{subfig}

\title{Encounter probability in the polymer Ring\\
Applications for polymer structure reconstruction using from chromosome capture data}
\author{Ofir Shukron}
\usetheme{Madrid}
\usecolortheme{dolphin}

\begin{document}
\begin{frame}
\titlepage
\end{frame}

\begin{frame}{Agenda}

\end{frame}

\begin{frame}{The Rouse chain}
\begin{enumerate}
\item The Rouse chain can be described as a sequence of $N$ beads connected by $N-1$ Hookian springs. 
\item the source of the Hookian behavior is derived from the change in the Helmholtz free energy of the chain while it undergoes stretching. 
\item in a homogeneous (in terms of spring constant) chains, the spring constant is defined as 
$k=\frac{3K_bT}{b^2}$
\item specifically, the end-to-end vector distribution is if we connect the two ends with a spring of constant $k=\frac{3K_bT}{(N-1)b^2}$
\item therefore we can find equivalence between the equilibrium distribution of chains and rings. 
\item The question is what is equivalence?
\end{enumerate}
\end{frame}

\begin{frame}{The configuration distribution}
\begin{enumerate}
\item Since each bond vector is independent, the configuration distribution is given by the multiplication of each bond probability.
\item closing the chain into a ring, adds a spring between bead 1 and N. 
\item 
\end{enumerate}

\end{frame}

\begin{frame}{Equivalence of Chains}
\begin{enumerate}
\item The nature of the Gaussian chain allows us to study one chain's properties in terms of others more simple.
\end{enumerate}

\end{frame}
\end{document}