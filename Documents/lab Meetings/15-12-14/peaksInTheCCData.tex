\documentclass[8pt]{beamer}
\usepackage{amsmath}
\usepackage{amssymb}
\usepackage{graphicx}
\usepackage{hyperref}
\usepackage{color}
\usepackage{float}
\usepackage{subfig}

\title{A method to extract peaks in chromosome conformation capture data}
\author{Ofir Shukron}
\usetheme{Madrid}
\usecolortheme{dolphin}
\begin{document}

\begin{frame}
\titlepage
\end{frame}

\begin{frame}{Motivation}
\begin{enumerate}
\item Multiple hypotheses are being tested on large amount of experimental data. 
\item A criteria to control the error rate is needed.
\item To control the error, one would like to reduce type I error. 
\item Traditional methods controlled the probability of at least one type I error, but were too restrictive.
\item New method of controlling the error called positive False Discovery Rate (pFDR) was developed\footnote{Storey JD. A direct approach to false discovery rates. J. R. Statist. Soc. B (2002)64, Part 3, pp. 479–498}. 
\item We want to apply this method to find frequent specific looping events in the chromosomes using chromosome capture (CC) data.
\end{enumerate}
\end{frame}

\begin{frame}{Background}
\begin{enumerate}
\item CC experiment capture millions of encounter events between different parts of the chromosome. 
\end{enumerate}
\end{frame}

\begin{frame}{Mathematical derivation}
Conducting $m$ hypothesis tests, using p-values, $P$, as our test statistics.\\
We fix a rejection region $\gamma=[0, \gamma]$, and we reject the null hypothesis $H$ is $P\leq\gamma$. $(\gamma>0)$\\
Let $V$ be the number of type I errors and $R$ the total number of rejections.\\
The pFDR is defined as: 
\begin{equation*}
pFDR=E\left(\frac{V}{R}| R>0 \right)
\end{equation*}

We assume that the null hypothesis $H$ is true ($H=0$) with an a priori probability $\pi_0$ and false ($H=1$) with probability $\pi_1$. We write (Storey 2001, Theorem 1)\\
\begin{equation*}
pFDR=\frac{\pi_0 Pr(P\leq\gamma|H=0)}{Pr(P\leq\gamma)}
\end{equation*}
By the Bayes rule 
\begin{equation*}
pFDR = Pr(H=0|P\leq\gamma)
\end{equation*}

Under the null hypothesis, the p-values are uniformly distributed. 
\begin{equation*}
pFDR=\frac{\pi_0\gamma}{Pr(P\leq\gamma)}
\end{equation*} 
\end{frame}

\begin{frame}{Mathematical derivation}
We now need an estimate of $\pi_0$ and $Pr(P\leq\gamma)$.\\ 
Let $R$ be the total rejected null hypotheses, and $W$ the total accepted hypotheses. 

\begin{equation*}
\hat{\pi_0}=\frac{\#(P_i>\lambda)}{(1-\lambda)m}=\frac{W(\lambda)}{(1-\lambda)m}, \qquad 0\leq\lambda< 1
\end{equation*}
We treat $\lambda$ as fixed in the following. 
\begin{equation*}
\hat{Pr}(P\leq\gamma)=\frac{R(\gamma)}{m}
\end{equation*}
Plugging in these estimates and remembering that the pFDR is a conditional probability measure, we have 
\begin{equation*}
p\hat{FDR}_\lambda (\gamma)=\frac{W(\lambda)\gamma}{(1-\lambda)R(\gamma)(1-(1-\gamma)^m)}
\end{equation*}
The equivalent of the p-values for the pFDR is called the q-value.\\
q-values are the minimum pFDR that can occur when rejecting a statistics with a value $t$. 
\begin{equation*}
q = \inf_{\{\gamma; t\in\Gamma\}}pFDR(\gamma)
\end{equation*}
\end{frame}

\begin{frame}{How do we do it in practice}

\begin{enumerate}
\item For the 5C data of $N$ encounter signals, $s_i(d)$, $i=1...N$. 
\item Calculate the background (expected) signal, $\mu(d)=\frac{1}{N_d}\sum_{i_d=1}^{N_d}(s_{i_d}(d))$, with $i_d$ the index of available observation in position $d$.
\item Calculate the background distribution $F_B(d)$. 
\item For each $d$ calculate the distribution, $F_d(z)$ of the z-score, $z_d(i_d)=\frac{s_{i_d}(d)-\mu(d)}{\sigma_d}$
\item Remark: if we are interested in the peaks, truncate the negative values of the z-scores.
\item For the rejection value $\gamma$ of the null distribution, calculate $P_d = F_d(F_B^{-1}(\gamma))$
\item Calculate the pFDR and the associated q-values, and set a threshold $\alpha$.
\item set the new threshold at $F_B^{-1}(\max\{P_d| q(P_d)<\alpha\})$
\end{enumerate}
\end{frame}

\begin{frame}{The 5C data}
\begin{figure}[H]
\includegraphics[scale=0.3]{tadDandENoraEtAl2012}
place image of 5c data-> encounter probability vs. distance-> image of signals -> 
\end{figure}

\end{frame}
\end{document}