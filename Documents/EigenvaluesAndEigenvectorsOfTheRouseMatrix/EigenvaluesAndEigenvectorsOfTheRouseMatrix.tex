\documentclass[12pt]{report}
\usepackage{amsmath}
\usepackage{amssymb}
\usepackage{graphicx}
%\usepackage{hyperref}
\usepackage{color}
\usepackage{float}
\begin{document}
\section{Eigenvalues of the Rouse Matrix}\label{section_eigenvaluesOfTheRouseMatrix}
In this document we derive the eigenvalues of the Rouse matrix. The Rouse matrix defines the connection in the harmonic potential of the Rouse polymer. The system of differential equation governing the dynamics of the chain comprised of $N$ monomers is 
\begin{equation}
\frac{d[X(t)]}{dt}=-k[R][X(t)]+[g(t)]
\end{equation}
where the $[.]$ notation represents a matrix (vector), $[g(t)]$ is an $N$ by 1 vector of normally distributed numbers with mean 0 and STD =1, $k$ is a constant and the matrix $[R]$ is defined as:
\begin{equation}
R=\left[
\begin{matrix}
 1 & -1 &  0 &  0 &...&  &  0 \\
-1 &  2 & -1 &  0 &...&  &  0 \\
 0 & -1 &  2 & -1 &...&  &  0 \\
 . &    &    &  . &   &  &  . \\
 . &    &    &  . &   &  &  . \\
 0 &    &    &    & -1& 2& -1 \\
 0 &    &    &    &  0&-1&  1 \\     
\end{matrix}
\right]
\end{equation}

To find the eigenvalues we calculate 
\begin{equation*}
D_N=\left|[R]-\lambda[I]\right|=0
\end{equation*}
which gives us a matrix of the form 
\begin{equation*}
R=\left[
\begin{matrix}
 y & -1 &  0 &  0 &...&  &  0 \\
-1 &  x & -1 &  0 &...&  &  0 \\
 0 & -1 &  x & -1 &...&  &  0 \\
 . &    &    &  . &   &  &  . \\
 . &    &    &  . &   &  &  . \\
 0 &    &    &    & -1& x& -1 \\
 0 &    &    &    &  0&-1&  y \\     
\end{matrix}
\right]
\end{equation*}
with $x=2-\lambda$, and $y=x-1$.

Developing the determinant by the last column we find a recursion relation as follows 
\begin{equation*}
D_N = yD_{N-1}-D_{N-2}
\end{equation*}

Since the recursion relation is slightly different for $D_{N-1}$, bt remains the same for all $j\leq N-1$, we solve the recursion relation for $D_{N-1}$ and then return to define the last term $D_N$ using the relation above. 

The recurrence relation is:
\begin{equation*}
D_z = xD_{z-1}-D_{z-2}
\end{equation*}

with the boundary conditions
\begin{equation*}
D_1 = y 
\end{equation*}
\begin{equation*}
D_2 = xy-1
\end{equation*}
We note that according to the Rouse matrix $x=y+1$.

The particular solution to the recursion relation is 
\begin{equation*}
D_z=e^{iz\theta}
\end{equation*}

substituting it into the recursion relation gives
\begin{equation*}
e^{iz\theta}=xe^{i(z-1)\theta}-e^{i(z-2)\theta}
\end{equation*}

hence
\begin{equation*}
x=2cos(\theta)
\end{equation*}

The general solution can then be defined as 
\begin{equation*}
D_z=Ae^{iz\theta}+Be^{-iz\theta}
\end{equation*}
where $A$ and $B$ are some constants to be defined.

 Using the boundary conditions we get 
 \begin{equation*}
 y = Ae^{i\theta}+Be^{-i\theta}
 \end{equation*}
 \begin{equation*}
 A=\frac{y-Be^{-i\theta}}{e^{i\theta}}=(2\cos(\theta)-1)e^{-i\theta}-Be^{-2i\theta} =\frac{1-e^{i\theta}}{e^{-i\theta}-e^{i\theta}}=e^{i\theta/2}\frac{e^{-i\theta/2}-e^{i\theta/2}}{-2i\sin(\theta)}=\frac{e^{i\theta/2}}{2\cos(\theta/2)}
 \end{equation*}
 and 
 \begin{equation*}
 B=\frac{e^{-i\theta}-1}{e^{-i\theta}-e^{i\theta}}=\frac{e^{-i\theta/2}}{2cos(\theta/2)}
 \end{equation*}
The general solution is now
\begin{equation}
D_z=\frac{e^{i\theta/2}}{2\cos(\theta/2)}e^{iz\theta}+\frac{e^{-i\theta/2}}{2\cos(\theta/2)}e^{-iz\theta}=\frac{\cos((z+1/2)\theta)}{\cos(\theta/2)}
\end{equation}
Since the determinant must vanish, we have 
\begin{equation*}
D_N = yD_{N-1}-D_{N-2}=0
\end{equation*}
substituting the expressions for $D_{N-1}$ and $D_{N-2}$ into the equation above yields 
\begin{equation*}
yD_{N-1}-D_{N-2} = D_1D_{N-1}-D_{N-2} = 
\end{equation*}
\begin{equation*}
\frac{\cos(3\theta/2)}{\cos(\theta/2)}\frac{\cos((N-1/2)\theta)}{\cos(\theta/2)}-\frac{\cos((N-3/2)\theta)}{\cos(\theta/2)}=0
\end{equation*}
therefore 
\begin{equation*}
\cos(3\theta/2)\cos((N-1/2)\theta)=\cos((N-3/2)\theta)\cos(\theta/2)
\end{equation*}
displaying the trigonometric functions as sum of exponentials we can get
\begin{equation*}
\cos((N+1)\theta)-\cos((N-1)\theta)=0
\end{equation*}
which gives 
\begin{equation*}
-2\sin(N\theta)\sin(\theta)=0
\end{equation*}
therefore 
\begin{equation*}
\theta = \frac{p\pi}{N}
\end{equation*}
$p=0,1...,N-1$ (since we have $N$ solutions)\\

The eigenvalues are then
\begin{equation*}
\lambda_p=2-x = 2-2\cos(\theta_p)=2\left(1-\cos(\frac{p\pi}{N})\right)=4\sin^2(\frac{p\pi}{2N})
\end{equation*}
$p=0,1,...,(N-1)$.

\section{Eigenvectors of the Rouse Matrix}\label{section_eigenvectorsOfTheRouseMatrix}
the $k^{th}$ entry in the $p^{th}$ eigenvector is 

\begin{equation*}
c_k = \sqrt{\frac{2}{N}}\sin(\frac{k\pi p}{N})
\end{equation*}

with $k=1,2,..,N$ and $p=0,1,...N-1$

\section{Eigenvalues of the Rouse Ring}\label{section_eigenvaluesRouseRing}
Connecting bead 1 and $N$ to form a ring, we now search for the eigenvalues of the new \textit{Rouse ring} matrix, $R$, harboring such a connection. For this end, we have to calculate the determinant of the matrix
\begin{equation*}
D_N=|R-\lambda I|=\left|
\begin{matrix}
 x  & -1 &  0 &  0 &...&   & -1 \\
-1  &  x & -1 &  0 &...&   &  0 \\
 0  & -1 &  x & -1 &...&   &  0 \\
 .  &    &    &  . &   &   &  . \\
  . &   .&    &    &  .&   & \\
 .  &    &    & -1 & x &-1 & 0 \\
 0  &    &    &    & -1& x & -1 \\
 -1 &   .&  . & .  &  0&-1 &  x \\     
\end{matrix}
\right|_{N\times N}
\end{equation*}

with $x=2-\lambda$. We can apply to the matrix of size $(N-1)\times (N-1)$ the same procedure as before and try to solve it recursively. The boundary conditions now read
\begin{equation*}
D_1 = x; D_2 = x^2-1
\end{equation*}
According to the solution in section \ref{section_eigenvaluesOfTheRouseMatrix} and in \cite{lin2011polymer}, we have 
\begin{equation*}
D_{N-1}= \frac{\sin(N\theta)}{\sin(\theta)}
\end{equation*}

The relationship between $D_N$ and $D_{N-1}$ is found by developing the determinant of $D_N$ according to the last column
\begin{equation*}
D_N = x(-1)^{2N}D_{N-1}+(-1)(-1)^{2N-1}D^*+(-1)(-1)^{N+1}D^{**}=xD_{N-1}+D^*+(-1)^{N+2}D^{**}
\end{equation*}
with the determinant $D^*$ and $D^{**}$ defined as the minors 
\begin{equation*}
D^*= 
\left| \begin{matrix}
 x  & -1 &  0 &  0  &... &  0 \\
-1  &  x & -1 &  0  &... &  0 \\
 0  & -1 &  x & -1  &... &    \\
 .  &    &    &  .  &    &    \\
 .  &    &    &  .  &    &    \\
 .  &    &    &  -1 &  x &-1  \\
 -1 &    &    &     &  0 &-1  \\   
\end{matrix}\right|_{(N-1)\times(N-1)} 
D^{**}=\left|\begin{matrix}
-1  &  x & -1 &  0  &...&  0 \\
 0  & -1 &  x & -1  &...&  0 \\
 .  &    &    &  .  &   &  . \\
 .  &    &    &  .  &   &  . \\
 .  &    &    &  -1 &  x& -1 \\
 0  &    &    &     & -1& x  \\
 -1 &    &    &     &  0&-1  \\   
\end{matrix} \right|_{(N-1)\times(N-1)}
\end{equation*}
We calculate the determinants $D^*$ and $D^{**}$ by minors according to the last row to get 
\begin{equation*}
D^* = (-1)(-1)^{2N-2}D_{N-2}+(-1)(-1)^{N-1+1}(-1)^{N-2}=(-1)^{2N-1}[D_{N-2}+1]
\end{equation*}

\begin{equation*}
D^{**}=(-1)(-1)^{2N-2}(-1)^{N-2} +(-1)(-1)^{1+N-1}D_{N-2}=(-1)^{3N-3}+(-1)^{N+1}D_{N-2}
\end{equation*}
Therefore, 
\begin{eqnarray*}
D_N & = & xD_{N-1} + (-1)^{2N-1}[D_{N-2}+1]+(-1)^{N+2}[(-1)^{3N-3}+(-1)^{N+1}D_{N-2}]\\
    & = & xD_{N-1} + (-1)^{2N-1}[D_{N-2}+1]+(-1)^{4N-1}+(-1)^{2N+3}D_{N-2}\\
    & = & xD_{N-1} + D_{N-2}[(-1)^{2N-1}+(-1)^{2N+3}]+(-1)^{2N-1}+(-1)^{4N-1}\\
    & = & xD_{N-1} - 2D_{N-2}-2
\end{eqnarray*}
Substituting the solution for $D_{N-1}$ and $D_{N-2}$, we get 
\begin{equation*}
D_N = \frac{x\sin(N\theta)-2\sin((N-1)\theta)-2\sin(\theta)}{\sin(\theta)}=0
\end{equation*}

Further simplification of $D_N$ leads to 
\begin{eqnarray*}
D_N &=& 2\frac{\cos(\theta)\sin(N\theta)-[\sin((N-1)\theta)+\sin(\theta)]}{\sin(\theta)}\\
    &=& \frac{2}{\sin(\theta)}[\cos(\theta)\sin(N\theta)-[2\sin(\frac{N\theta}{2})\cos(\frac{(N-2)\theta}{2})]\\
    &=& \frac{4}{\sin(\theta)}[\cos(\theta)\sin(\frac{N\theta}{2})\cos(\frac{N\theta}{2})-\sin(\frac{N\theta}{2})[\cos(\frac{N\theta}{2})\cos(\theta)+\sin(\frac{N\theta}{2})\sin(\theta)]\\
    &=& -4\sin^2(\frac{N\theta}{2})
\end{eqnarray*}

Equating $D_N=0$ we get the solutions
\begin{equation*}
\theta_p=\frac{2\pi p}{N}
\end{equation*}
for $p=0,1,...,(N-1)$

Therefore, the eigenvalues of the Rouse ring are 
\begin{equation*}
\lambda_p=2(1-cos(\theta_p))= 2(1-cos(2\pi p/N))
\end{equation*}
Since $\cos(\frac{2\pi \phi}{N})=\cos(\frac{2\pi(N-\phi)}{N})$, we have eigenvalues multiplicities. In our case, setting $\phi=p$ we have $\cos(\frac{2\pi p}{N})=\cos(\frac{2\pi(N-p)}{N})$ for $p=1,2,3,..,(N-2)$. For the end-values, $p=0$ and $p=N-1$ the eigenvalues are unique. 



\section{Eigenvectors of the Rouse Ring}\label{eigenvectorsOfTheRouseRing}
\subsection{eigenvalues with multiplicity 1}\label{subsection_eigenvaluesWithMultiplicity1}
The eigenvectors corresponding to eigenvalues with algebraic multiplicity 1. From the relation $Rv=\lambda v$ where $v$  is an eigenvector.
We have the recursion relation for the component of the $v$ vector, $j=2,...,N-1$
\begin{equation*}
v_j = \frac{v_{j-1}+v_{j+1}}{2\cos(\theta)}
\end{equation*}
The boundary conditions are 
\begin{equation*}
v_1 = \frac{v_2+v_N}{2\cos(\theta)},\quad v_N = \frac{v_1+v_{N-1}}{2\cos(\theta)}
\end{equation*}
Rearranging the terms for $v_1$ and $v_N$
\begin{equation*}
v_1 = \frac{2v_2\cos(\theta)+v_{N-1}}{4\cos^2(\theta)-1}\qquad v_N = \frac{v_2+2v_{N-1}\cos(\theta)}{4\cos^2(\theta)-1}
\end{equation*}
where, again, we have used the relation $2-\lambda = 2\cos(\theta)$, resulting from substituting the particular solution $v_j=e^{ij\theta}$ into the recursion relation for $v_j$. 

Substituting the general solution into the first boundary condition we get 
\begin{equation*}
A(e^{i\theta}(4\cos^2(\theta)-1)-2\cos(\theta)e^{2i\theta}-e^{i(N-1)\theta})=-B(e^{-i\theta}(4\cos^2(\theta)-1)-2\cos(\theta)e^{-2i\theta}-e^{-i(N-1)\theta})
\end{equation*}
if we set 
\begin{equation*}
f(\theta) = (e^{i\theta}(4\cos^2(\theta)-1)-2\cos(\theta)e^{2i\theta}-e^{i(N-1)\theta})
\end{equation*}
then 
\begin{equation*}
Af(\theta)=-Bf(-\theta)
\end{equation*}
notice that 
\begin{equation*}
f(\theta)=\left<[4\cos^2(\theta)-1,2\cos(\theta),1],-[e^{i\theta},e^{2i\theta},1] \right>
\end{equation*}
and 
\begin{equation*}
f(-\theta)=\left<[4\cos^2(\theta)-1,2\cos(\theta),1],-[e^{-i\theta},e^{-2i\theta},1] \right>
\end{equation*}
The vectors $[e^{-i\theta},e^{-2i\theta},1]$ and $[e^{i\theta},e^{2i\theta},1]$ are opposite in directions but otherwise equal. Indeed 
\begin{equation*}
 cos(\phi)=\frac{\left<[e^{-i\theta},e^{-2i\theta},1],[e^{i\theta},e^{2i\theta},1]\right>}{\sqrt{3}\sqrt{3}}=\frac{3}{\sqrt{3}\sqrt{3}}\Rightarrow \phi = k\pi \qquad k=0,1,2,....
\end{equation*}
with $\phi$ the angle between vectors. Hence, we conclude $f(\theta)=-f(-\theta)$ and therefore 
\begin{equation*}
A=B
\end{equation*}
and the general solution becomes 
\begin{equation*}
v_j = A(e^{ij\theta}+e^{-ij\theta})=2A\cos(j\theta)
\end{equation*}
The values of $v_1$ and $v_N$ can be written as 
\begin{equation*}
v_1 = \frac{2A[2\cos(2\theta)\cos(\theta)+\cos((N-1)\theta)]}{4\cos^2(\theta)-1} \qquad v_N = \frac{2A[\cos(2\theta)+2\cos(\theta)\cos((N-1)\theta)]}{4\cos^2(\theta)-1}
\end{equation*}

Using the normalization conditions, namely $\sum_{j=1}^{N}v_j^2 =1$, we write 
\begin{equation*}
\sum_{j=2}^{N-1}A^2\cos^2(j\theta) +v_1^2+v_N^2=1
\end{equation*}
and therefore 



\begin{eqnarray*}
A &=& (\sum_{j=2}^{N-1}\cos^2(j\theta) +\left(\frac{2[2\cos(2\theta)\cos(\theta)+\cos((N-1)\theta)]}{4\cos^2(\theta)-1}\right)^2\\   &+&\left(\frac{2[\cos(2\theta)+2\cos(\theta)\cos((N-1)\theta)]}{4\cos^2(\theta)-1}\right)^2 )^{-0.5} = B
\end{eqnarray*}
  
 \subsection{eigenvectors with multiplicity 2}\label{subsection_eigenvaluesWithMultiplicity2} 
% Bibliography  
\bibliographystyle{plain}
\bibliography{EigenvaluesAndEigenvectorsOfTheRouseMatrixBibliography} % the bibliography.bib
\end{document}