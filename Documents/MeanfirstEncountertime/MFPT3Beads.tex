\documentclass{report}
\usepackage{amsmath}
\usepackage{amssymb}
\usepackage{graphicx}
%\usepackage{hyperref}
\usepackage{color}
\title{Simultaneous Mean First Encounter Time for 3 beads of the Rouse chain}
\begin{document}
	\maketitle
 Given a Rouse chain, composed of $N$ beads, let the position of any bead be $R_i(t)$ , $i=1..N$. 	We wish to calculate the mean first simultaneous encounter time (MFET) of 3 beads $1,N',N$, with $1<N'<N$
  
	\begin{equation}\label{eq_encounterCondition}
	\tau=inf \{t ; |R_1(t)-R_N(t)|< \epsilon , |R_1(2)-R_{N'}(t)|<\epsilon \}
	\end{equation}
	Given the harmonic potential induced by the springs in the Rouse chain 
	\begin{equation}
	\phi(R_1,R_2,...,R_N)_{Rouse}=\frac{2}{\kappa}\sum_{n=1}^N (R_n-R_{n-1})^2
	\end{equation}
	with $\kappa$ the spring constant, the Rouse system is defined as 
	\begin{equation}\label{eq_RouseSystem}
	\dot{R}_n(t) = -D\nabla_{R_n}\phi_{Rouse} +\sqrt{2D}\dot{w}_n(t)
	\end{equation}

 We transform the Rouse system (\ref{eq_RouseSystem}) and the condition \ref{eq_encounterCondition} to their normal coordinates representation.
 \begin{equation}\label{eq_normalTransform}
 u_p(t)=\sum_{n=1}^N \alpha ^n_p R_n(t)
 \end{equation} 
 where 
 \begin{equation}\label{eq_eigenValues}
 \alpha_p^n = \begin{cases}
\sqrt{\frac{1}{N}}, & p=0 \\
\sqrt{\frac{2}{N}}\cos((n-1/2)\frac{p\pi}{N})  & otherwise
 \end{cases}
 \end{equation}
 the backward transformation is defined as 
 \begin{equation}\label{eq_backTransform}
 R_n(t) =\sum_{p=0}^{N-1}\alpha_p^n u_p(t)
 \end{equation}
 	The rouse system (\ref{eq_RouseSystem}) is now represented as 
 	
 \begin{equation}\label{eq_RouseSystemNormalCoordinates}
 \dot{u}_p(t)=-D_p\kappa_pu_p(t)+\sqrt{2D_p}\dot{w}_p(t)
 \end{equation}
 
 with $D_p=D, \kappa_p = 4\kappa \sin^2(p\pi /2N), p=1,2,..,N-1$
 
The simultaneous encounter condition (\ref{eq_encounterCondition}) can now be defined in terms of $u_p$. The conditions $|R_1(t)-R_N(t)|<\epsilon$, and  $|R_1(t)-R_{N'}(t)|<\epsilon$

\begin{equation}\label{eq_encounterConditionNormalCoordinates1}
|\sum_{p=0}^{N-1}\alpha_p^1 u_p(t)-\alpha_p^Nu_p(t)|=|\sum_{p=0}^{N-1}(\alpha_p^1-\alpha_p^N)u_p(t)|=0
\end{equation}
\begin{equation}\label{eq_encounterConditionNormalCoordinates2}
|\sum_{p=0}^{N-1}\alpha_p^1 u_p(t)-\alpha_p^{N'}u_p(t)| =|\sum_{p=0}^{N-1}(\alpha_p^1-\alpha_p^{N'})u_p(t)|=0
\end{equation}

substituting (\ref{eq_eigenValues}) into (\ref{eq_encounterConditionNormalCoordinates1}), and (\ref{eq_encounterConditionNormalCoordinates2}) we find 
\begin{equation}
\sum_{p=0}^{N-1}\sqrt{\frac{2}{N}}(\cos(0.5p\pi /N)-\cos((N'-0.5)p\pi /N))u_p(t) = 0
\end{equation}


\end{document}